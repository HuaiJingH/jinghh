\documentclass{ctexart}

\usepackage{graphicx}
\usepackage{amsmath}

\title{作业一: 洛必达法则的叙述与证明}


\author{黄槐靖\\ 专业:信息与计算科学  学号:3210105250}

\begin{document}

\maketitle
这是一个来自数学领域的问题。
\section{问题描述}
问题叙述如下: 设函数$f(x)$和$g(x)$在$(a,a+d]$上可导(d是某个正常数),且$g'(x) \neq 0$.若此时有
 \begin{equation}
 \lim_{x \to a+}f(x)=\lim_{x \to a+}g(x)=0\notag
 \end{equation}
或
 \begin{equation}
 \lim_{x \to a+}g(x)=\infty\notag
 \end{equation}
且$\lim\limits_{x \to a+}\frac{f'(x)}{g'(x)}$存在(可以是有限数或$\infty$),则成立
 \begin{equation}
 \lim_{x \to a+}\frac{f(x)}{g(x)}=\lim_{x \to a+}\frac{f'(x)}{g'(x)}\notag.
 \end{equation}

\section{证明}
证明:这里仅对$\lim\limits_{x \to a+}\frac{f'(X)}{g'(X)}=A$为有限数时来证明,当$A$为无穷大时的证明过程是类似的。\\
       先证明$\lim_{x \to a+}f(x)=\lim_{x \to a+}g(x)=0$的情况。\\
      由于函数在$x=a$处的值与$x \to a+$时的极限无关,因此可以补充定义
   \begin{equation}
   f(a)=g(a)=0\notag
   \end{equation}
      使得$f(x)$和$g(x)$在$(a,a+d]$上连续。这样,经补充定义后的函数$g(x)$和$g(x)$在$(a,a+d]$上满足Cauchy中值定理的条件,因而对于任意$x \in (a,a+d)$,存在$\xi \in (a,a+d)$,满足
    \begin{equation}
    \frac{f(x)}{g(x)}=\frac{f(x)-f(a)}{g(x)-g(a)}=\frac{f'(\xi)}{g'(\xi)}.
    \end{equation}
        当$x \to a+$时显然有$\xi \to a+$.由于$\lim\limits_{x \to a+}\frac{f'(x)}{g'(x)}$存在,在两端令$x \to a+$,即有
    \begin{equation}
    \lim_{x \to a+}\frac{f(x)}{g(x)}=\lim_{\xi \to a+}\frac{f'(\xi)}{g'(\xi)}=\lim_{x \to a+}\frac{f'(x)}{g'(x)}.
    \end{equation}
        下面证明$\lim\limits_{x \to a+}=\infty$时的情况。
        记$x_{0}$是$(a,a+d]$中任意一个固定点,则当$x  \neq x_{0}$时,$\frac{f(x)}{g(x)}$可以改写为
     \begin{align}
     \frac{f(x)}{g(x)}&=\frac{f(x)-f(x_{0})}{g(x)}+\frac{f(x_{0})}{g(x)}\notag\\
     &=\frac{g(x)-g(x_{0})}{g(x)}\times\frac{f(x)-f(x_{0})}{g(x)-g(x_{0})}+\frac{f(x_{0})}{g(x)}\notag\\
     &=[1-\frac{g(x_{0})}{g(x)}]\times\frac{f(x)-f(x_{0})}{g(x)-g(x_{0})}+\frac{f(x_{0})}{g(x)}.\label{sec:1}
     \end{align}
         由(\ref{sec:1})得,
     \begin{align}
     |\frac{f(x)}{g(x)}-A|&=|[1-\frac{g(x_{0})}{g(x)}]\times\frac{f(x)-f(x_{0})}{g(x)-g(x_{0})}+\frac{f(x_{0})}{g(x)}-A|\notag\\
     &\le|1-\frac{g(x_{0})}{g(x)}|\times|\frac{f(x)-f(x_{0})}{g(x)-g(x_{0})}-A|+|\frac{f(x_{0}-Ag(x_{0}))}{g(x)}|.
     \end{align}
          因为$\lim\limits_{x \to a+}\frac{f'(x)}{g'(x)}=A$,所以对于任意$\epsilon>0$,存在$\rho>0(\rho<d)$,当$0<x-a<\rho$时,
          \begin{equation}
          |\frac{f'(x)}{g'(x)}-A|<\epsilon
          \end{equation}
          取$x_{0}=a+\rho$,有Cauchy中值定理,对于任意$x \in (a,x_{0})$,存在$\xi \in (x,x_{0}) \subset (a,a+\rho)$,满足
          \begin{equation}
          \frac{f(x)-f(x_{0})}{g(x)-g(x_{0})}=\frac{f'(\xi)}{g'(\xi)}.\label{sec:2}
          \end{equation}
          由(\ref{sec:2})得到
          \begin{equation}
          |\frac{f(x)-f(x_{0})}{g(x)-g(x_{0})}-A|=|\frac{f'(\xi)}{g'(\xi)}-A|<\epsilon.
          \end{equation}
          又因为$\lim\limits_{x \to a+}g(x)=\infty$,所以可以找到正数$\delta<\rho$,当$0<x-a<\delta$时,成立
          \begin{equation}
          |1-\frac{g(x_{0})}{g(x)}|<2,|\frac{f(x_{0})-Ag(x_{0})}{g(x)}|<\epsilon.
          \end{equation}
          综上所述,即知对于任意$\epsilon>0$,存在$\delta>0$,当$0<x-a<\delta$时,
          \begin{equation}
          |\frac{f(x)}{g(x)}-A|\le|1-\frac{g(x_{0})}{g(x)}|\times|\frac{f(x)-f(x_{0})}{g(x)-g(x_{0})}|+|\frac{f(x_{0})-Ag(x_{0})}{g(x)}|<3\epsilon,
          \end{equation}
          由定义即得
           \begin{equation}
           \lim_{x \to a+}\frac{f(x)}{g(x)}=\lim_{x \to a+}\frac{f'(x)}{g'(x)}\notag.
           \end{equation}




\end{document}
